(a) Motivation of the project \linebreak[1]
(b) Aims of the project, ideal place first paragraph \linebreak[1]
(c) Structure of the report 

\section{old intro}
Inspired by the ideas of using blockchain to benefit society
through decentralised applications in [5], this project seeks to
design and develop a distributed solution to aid current
immunisation information systems. As mentioned in [6],
current implementations of IISs, especially that in developing
countries, are lacking in terms of technological proficiency.
As stated in [7], IISs are centralised repositories of personally
identifiable vaccination information for individual members of
a served population. This project aims to produce a
decentralised version of an IIS, utilising Decentralised Ledger Technology (DLT) 
which has been highly discussed since the wide adoption of blockchain technology introduced by \cite{nakamoto_bitcoin_2019} - As discussed in \cite{sunyaev_distributed_2020}.
From research such as [9] and [10], we know that blockchain provides the
required security and privacy necessary for implementing
these types of systems. [10] states that blockchain technology
can reform the interoperability of healthcare databases.
The system this project proposes uses a permissioned
blockchain, in which the health authorities of different nations
are trusted nodes in the network. Each health authority will
provide immunisation records of their population to the ledger
– such data will only be accessible by the health authority that
provided it as well as the individual the record belongs to.
This shall be enforced by using asymmetrical cryptography. A
private key will be required to access these records, the key
being held by the individual and their relevant health
authority. This enables the individual’s control over their
immunisation records along with the ability to provide their
private key and verify their immunisations. This, as discussed
in [7], will simplify the processes of vaccination verification
that may be required in pandemic scenarios, registering for
school, starting with a new employer, or crossing international
borders. This system would benefit bodies striving to verify
individuals’ immunisations internationally.
This report explores the required components of such a
system, the professional considerations that are necessary for
building this system such as legislation and ethical concerns,
along with the requirements of an IIS and that of one
implemented using blockchain.

Forgery and personal data security are dominant concerns of similar projects, but such problems are routinely solved for financial and other sensitive transactions. \cite{dye_covid-19_2021}
This project attempts to utilise the techniques that achieve this, in other industries, to the vaccination passport problem. % put it in the intro <-------------------------------------------------------------!!!

As divulged in \cite{sunyaev_distributed_2020}, DLT designs can be instantiated as a \emph{public} or \emph{private} distributed ledger \cite{yeow_decentralized_2018}, \cite{xu_taxonomy_2017}.
In public DLT designs, the underlying network allows arbitrary nodes to join and participate in the distributed ledger's maintenance. No registration or verification of the nodes' identities is required.
Public DLT designs are often maintained by a large number of nodes, like with Bitcoin and Ethereum. The designs enable consistent high levels of availability. 
To allow for a high number of (arbitrary) nodes to find consensus, the designs for public DLT should be well scalable to not deter performance as more nodes join the network. \cite{sunyaev_distributed_2020}

In contrast, private DLT designs engage a defined set of nodes, with each node identifiable and known to the other network nodes. This means that private DLT designs require node verification when joining the distributed ledger, for example, by using Public Key Infrastructure (PKI).
PKI comprises hardware, sfotware, policies, procedures and roles that are used for the secure electronic transfer of data by means of an insecfure network. A PKI manages the creation, distribution and revocation of digital certificates, which the use of public key cryptography requires. \cite{sunyaev_distributed_2020}

Blockchains can execute programmable transaction logic in the form of \emph{smart contracts} as demonstrated by Ethereum \cite{noauthor_ethereum_nodate}.
The predacessor of smart contracts were the scripts in Bitcoin, introduced in \cite{nakamoto_bitcoin_2019}.
Smart contracts function as \emph{trusted distributed applications} and gains its security from teh blockchain and the underlying consensus among peers.
This is similar to the approach of building resilient applications with state-machine replication (SMR) \cite{schneider_implementing_1990}.
Though, blockchains are different from traditional SMR with Byzantine 

\section{Problem Space, blockchain intro } % get a good title

\subsection{Blockchain architecture cryptography??}
% where should this cryptology stuff go? Near the start? more detailed part near start, maybe a whole chapter on blockchain inc architecture / use cases? etc 
Randomized hashing offers the signer additional protection by reducing the likelihood that a preparer can generate two or more messages 
that ultimately yield the same hash value during the digital signature generation process – even if it is practical to find collisions for the hash function. 

hash functions are formally defined in \cite{rompay_analysis_nodate} as $a function h: D --> R where the domain D = {0,1}*, and the range R = {0,1}^n for smoe n >= 1 $
using one way hash functions, as defined by Merkle in \cite{merkle_secrecy_1979}, are hash functions

store only hashes, this way data is kept anonymous and GDPR isn't a worry. Also, keeps the private data away from any verifiers etc because all they check against is a hash. % ref for why hashes are irreversible

digital signatures  $<--$ NEW!! NIST Publishes Special Publication 800-106 (Draft) "Randomized Hashing Digital Signatures". that documents our basic randomization technique for use with digital signatures. \url{http://csrc.nist.gov/publications/drafts/Draft-SP-800-106/Draft-SP800-106.pdf}
