
 The system proposed in this project will process personal data,
 in the form of immunisation records, so necessitates
 consideration of ethical and legal requirements.

 \section{BCS Code of Conduct}
 This project has been aligned with the BCS Code of Conduct;
 relevant sections are as follows:
 \renewcommand{\labelenumii}{\alph{enumii}}
  \begin{enumerate}
    \item Public Interest
    You shall:
    \begin{enumerate}
      \item have a due regard for public health, privacy,
      security and wellbeing of others and the environment.
      Encryption methods will be used where necessary to
      ensure the confidentiality of information in the
      system.
      \item have due regard for the legitimate rights of Third
      Parties.
      This system will make use of asymmetric-key
      encryption as well as hash functions to protect data,
      including that of third parties.
      \item promote equal access to the benefits of IT and
      seek to promote the inclusion of all sectors in society
      wherever opportunities arise.
      This project proposes the system detailed be adopted
      by nations to provide a means of access to personal
      immunisation records, enabling ease of access for the
      population.
     \end{enumerate}
    \item Professional Competence and Integrity
    You shall:
    \begin{enumerate}
     \item only undertake to do work or provide a service
     that is within your professional competence.
     \item NOT claim any level of competence that you do
     not possess. This project has been thoroughly considered and the
     conclusion has been reached that it is within my
     professional competence.
     \item develop your professional knowledge, skills and
     competence on a continuing basis. Maintaining
     awareness of technological developments,
     procedures, and standards that are relevant to your
     field.
     This project explores the current solutions for
     immunisation information systems, evolving my
     professional knowledge. This project displays an
     awareness of technological standards and procedures necessary for a distributed IIS.
     \item ensure that you have the knowledge and
     understanding of Legislation and that you comply
     with such Legislation, in carrying out your
     professional responsibilities.
     This project includes a section exploring the
     legislation concerning this system. Specifically, the
     geographic area in which this system is being
     produced and how legislation governs the operations
     of such a system.
     \item respect and value alternative viewpoints and,
     seek, accept and offer honest criticisms of work.
     This project shall regularly be shared with my project
     supervisor to gain alternative viewpoints and
     criticisms, which will be implemented.
     \item avoid injuring others, their property, reputation,
     or employment by false or malicious or negligent
     action or inaction.
     The Background Information section of the project details the
     necessary security methods to maintain
     confidentiality, integrity and authenticity in the
     system. 
     Briefly, the sytem utilises various techiques such as hashing data, Public Key Infrastructure and anonymous data verification to maintain confidentiality when handling Personally Idetifiable Information (PII).
  \end{enumerate}
  \end{enumerate}
 
  \subsection{Legislation}
  The General Data Protection Regulation (GDPR) is a privacy
  and security law, passed by the European Union (EU),
  imposes obligations onto organizations that target or collect
  data related to EU citizens and residents. \cite{noauthor_general_nodate} \linebreak[1]
 
 The GDPR sets out several key principles:
 \begin{enumerate}
   \item Lawfulness, fairness and transparency
   \item Purpose limitation
   \item Data minimization
   \item Accuracy
   \item Storage limitation
   \item Integrity and confidentiality
   \item Accountability
 \end{enumerate}
 
 These stated principles are essential to our approach to
 processing personal data. Compliance with the principles
 established in the GDPR is fundamental to ensuring good
 practice in data protection.
 In Article 35(1) of the GDPR it states that a Data Protection
 Impact Assessment (DPIA) is required “Where a type of
 processing in particular using new technologies, and taking
 into account the nature, scope, context and purposes of the
 processing, is likely to result in a high risk to the rights and
 freedoms of natural persons, the controller shall, prior to the
 processing, carry out an assessment of the impact of the
 envisaged processing operations on the protection of personal
 data. A single assessment may address a set of similar
 processing operations that present similar high risks.”. This
 asserts that if the system being designed and developed in this
 project is implemented in the real world performing a DPIA is
 mandatory. Though, a DPIA will not be necessary for this
 undertaking.
 As the GDPR is mainly concerned with the European
 Economic Area (EEA), producing this system in the UK
 brings concerns. The UK is currently in a transition period
 until the 31st of December 2020. At the end of this transition
 period the UK will become a third country. Presently, the UK
 is seeking adequacy decisions from the European
 Commission. “The effect of an adequacy decision is that
 personal data can be sent from an EEA state to a third country
 without any further safeguard being necessary” because “The
 European Commission has the power to determine whether a
 third country has an adequate level of data protection.” \cite{noauthor_information_2020}. If
 the adequacy decision is not secured, by the end of the
 transition period, the provisions set out in \cite{noauthor_agreement_2019} will take effect.
 
 \subsection{Addressing Considerations}
 Because of Art. 17, GDPRs 'right to erasure' \cite{noauthor_art_nodate}, 
 The system has implemented chaincode to delete assets held on the network.

 Medical records are pieces of personal, sensitive data which are stored, processed, processed and transmitted in healthcare systems.
Though, recital 26 is note applicable to anonymous data, maintaing data anonymity is how compliance with GDPR shall be maintained. \footnote{https://gdpr-info.eu/recitals/no-26/}



 Randomized hashing offers the signer additional protection by reducing the likelihood that a preparer can generate two or more messages 
that ultimately yield the same hash value during the digital signature generation process – even if it is practical to find collisions for the hash function. 

Hash functions are formally defined in \cite{rompay_analysis_nodate} as a function \begin{math} h: D \rightarrow R \end{math} where the domain \begin{math} D = \{0,1\}^* \end{math} and the range \begin{math} R = \{0,1\}^n \end{math} for some \begin{math} n \geq 1 \end{math}
using one way hash functions, as defined by Merkle in \cite{merkle_secrecy_1979}, are hash functions

Storing only hashes in the database, keeps data anonymous and is complacent with the GDPR. 
Also, this keeps the private data away from any verifiers etc, as all they check against is a hash.

"Issues of concern are falsified or counterfeit vaccine certificates" are described in \cite{schlagenhauf_variants_2021}, when considering digital vaccination passports. 
Blockchain enables an immutable store of data, rendering any attempt at providing a spoofed record would be nullified due to the data the individual provides not being in the ledger.
