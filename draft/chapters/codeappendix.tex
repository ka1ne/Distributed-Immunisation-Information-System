\section{Web Application Code}
The full code for the web application developed for the project; uses Node, ExpressJS, MongoDB. \footnote{Aided by official documentation for Hyperledger Fabric "https://hyperledger-fabric.readthedocs.io/en/latest/write\_first\_app.html"} \label{appendix:webapp}
\begin{lstlisting}[caption={Web application for DIIS.}]

\end{lstlisting}


\subsection{db.js}
\label{appendix:dbjs}
\begin{lstlisting}
    const mongoose = require('mongoose');

    // require('dotenv').config()
    
    /*
    const {
        MONGO_USERNAME,
        MONGO_PASSWORD,
        MONGO_HOSTNAME,
        MONGO_PORT,
        MONGO_DB
    } = process.env;
    */
    
    const MONGO_USERNAME = 'user';
    const MONGO_PASSWORD = 'userpassword123';
    const MONGO_HOSTNAME = '127.0.0.1';
    const MONGO_PORT = '27017';
    const MONGO_DB = 'diis';
    
    const options = {
        useNewUrlParser: true,
        reconnectTries: Number.MAX_VALUE,
        reconnectInterval: 500,
        connectTimeoutMS: 10000,
    };
    
    const url = `mongodb://${MONGO_USERNAME}:${MONGO_PASSWORD}@${MONGO_HOSTNAME}:${MONGO_PORT}/${MONGO_DB}?authSource=admin`;
    
    mongoose.connect(url, options).then(function () {
        console.log('MongoDB is connected');
    })
        .catch(function (err) {
            console.log(err);
        });
\end{lstlisting}


\subsection{app.js}
\label{appendix:appjs}
\begin{lstlisting}
    const express = require('express');
    const app = express();
    const router = express.Router();
    const db = require('./db');
    const recordRouter = require('./routes/records');
    
    const path = __dirname + '/views/';
    const port = 3000;
    
    app.engine('html', require('ejs').renderFile);
    app.set('view engine', 'html');
    app.use(express.urlencoded({ extended: true }));
    app.use(express.static(path));
    app.use('/records', recordRouter);
    
    app.listen(port, function () {
      console.log('Example app listening on port 3000!')
    })
\end{lstlisting}


\subsection{Routes}
\subsubsection{index.js}
\label{appendix:indexroute}
\begin{lstlisting}
    const express = require('express');
    const router = express.Router();
    const path = require('path');
    
    router.use (function (req,res,next) {
      console.log('/' + req.method);
      next();
    });
    
    router.get('/',function(req,res){
      res.sendFile(path.resolve('views/index.html'));
    });
    
    module.exports = router;    
\end{lstlisting}

\subsubsection{records.js}
\label{appendix:recordroute}
\begin{lstlisting}
    const express = require('express');
    const router = express.Router();
    const record = require('../controllers/records');
    
    router.get('/', function(req, res){
        record.index(req,res);
    });
    
    router.post('/addrecord', function(req, res) {
        record.create(req,res);
    });
    
    router.get('/genuuid', function(req, res) {
        record.createuuid(req,res);
    });
    
    router.post('/checkrecord', function(req, res) {
        record.check(req,res);
    });
    
    module.exports = router;     
\end{lstlisting}

\subsection{Views} 
\label{appendix:views}
This section of the appendix displays the code used to produce the client views, using a template engine. 
\subsubsection{index.html}
\begin{lstlisting}
    <!DOCTYPE html>
    <body>
        <a href="/records" role="button">Records</a>
      </body>
    </html>
    
\end{lstlisting}
\subsubsection{records.html}
\begin{lstlisting}
    <!DOCTYPE html>
    <form action="/records/genuuid" method="get">
        <button type="submit">Get UUID</button>
    </form>
    <form action="/records/addrecord" method="post"> 
        <div class="caption">Enter Your Record</div>
        <input type="text" placeholder="Record UUID" name="uuid" <%=records[i].uuid; %>
        <input type="date" placeholder="Record timestamp" name="timestamp" <%=records[i].timestamp; %>
        <input type="text" placeholder="Record owner" name="owner" <%=records[i].owner; %>
        <input type="text" placeholder="Record expiration" name="exp" <%=records[i].exp; %>
        <button type="submit">Submit</button>
    </form>
    
    <form action="/records/checkrecord" method="post"> 
        <div class="caption">Enter The Record UUID</div>
        <input type="text" placeholder="Record UUID" name="uuid" <%=records[i].uuid; %>
        <button type="submit">Submit</button>
    </form>
    </html>    
\end{lstlisting}


\subsection{records.js}
The following code is that of the controller for the application. \label{appendix:controller}
\begin{lstlisting}
  'use strict';

  const path = require('path');
  const Record = require('../models/records');
  const uuid4 = require('uuid').v4;
  
  const {
      createHash,
    } = require('crypto');
  
  const hash = createHash('sha256');
  
  const { Gateway, Wallets } = require('fabric-network');
  const FabricCAServices = require('fabric-ca-client');
  const { buildCAClient, registerAndEnrollUser, enrollAdmin } = require('../utils/CAUtil.js');
  const { buildCCPOrg1, buildWallet, buildCCPOrg2 } = require('../utils/AppUtil.js');
  
  const channelName = 'mychannel';
  const chaincodeName = 'record';
  const mspOrg1 = 'Org1MSP';
  const mspOrg2 = 'Org2MSP';
  const walletPath = path.join(__dirname, '..', 'wallet');
  const org1UserId = 'appUser';
  const org2UserId = 'verifier';
  
  function prettyJSONString(inputString) {
      return JSON.stringify(JSON.parse(inputString), null, 2);
  }
  
  /**
    * @param record A record produced using the body of the request in the create function
    */
  async function submit(record) {
      try {
          // build an in memory object with the network configuration (also known as a connection profile)
          const ccp = buildCCPOrg1();
  
          // build an instance of the fabric ca services client based on
          // the information in the network configuration
          const caClient = buildCAClient(FabricCAServices, ccp, 'ca.org1.example.com');
  
          // setup the wallet to hold the credentials of the application user
          const wallet = await buildWallet(Wallets, walletPath);
  
          // in a real application this would be done on an administrative flow, and only once
          await enrollAdmin(caClient, wallet, mspOrg1);
  
          // in a real application this would be done only when a new user was required to be added
          // and would be part of an administrative flow
          await registerAndEnrollUser(caClient, wallet, mspOrg1, org1UserId, 'org1.department1');
  
          // Create a new gateway instance for interacting with the fabric network.
          // In a real application this would be done as the backend server session is setup for
          // a user that has been verified.
          const gateway = new Gateway();
  
          try {
              // setup the gateway instance
              // The user will now be able to create connections to the fabric network and be able to
              // submit transactions and query. All transactions submitted by this gateway will be
              // signed by this user using the credentials stored in the wallet.
              await gateway.connect(ccp, {
                  wallet,
                  identity: org1UserId,
                  discovery: { enabled: true, asLocalhost: true } // using asLocalhost as this gateway is using a fabric network deployed locally
              });
  
              // Build a network instance based on the channel where the smart contract is deployed
              const network = await gateway.getNetwork(channelName);
  
              // Get the contract from the network.
              const contract = network.getContract(chaincodeName);
  
              // Initialize a set of asset data on the channel using the chaincode 'InitLedger' function.
              // This type of transaction would only be run once by an application the first time it was started after it
              // deployed the first time. Any updates to the chaincode deployed later would likely not need to run
              // an "init" type function.
              console.log('\n--> Submit Transaction: InitLedger, function creates the initial set of assets on the ledger');
              await contract.submitTransaction('InitLedger');
              console.log('*** Result: committed');
  
              // Let's try a query type operation (function).
              // This will be sent to just one peer and the results will be shown.
              console.log('\n--> Evaluate Transaction: GetAllAssets, function returns all the current assets on the ledger');
              let result = await contract.evaluateTransaction('GetAllAssets');
              console.log(`*** Result: ${prettyJSONString(result.toString())}`);
  
              var stamp = Date.parse(record.timestamp)
              var exp = Date.parse(record.exp)
  
              // Now let's try to submit a transaction.
              // This will be sent to both peers and if both peers endorse the transaction, the endorsed proposal will be sent
              // to the orderer to be committed by each of the peer's to the channel ledger.
              console.log('\n--> Submit Transaction: CreateAsset, creates new asset with UUID, timestamp, owner and expiration arguments');
              result = await contract.submitTransaction('CreateAsset', record.uuid, stamp, record.owner, exp);
              console.log('*** Result: committed');
              if (`${result}` !== '') {
                  console.log(`*** Result: ${prettyJSONString(result.toString())}`);
              }
  
              console.log('\n--> Evaluate Transaction: ReadAsset, function returns an asset with a given assetID');
              result = await contract.evaluateTransaction('ReadAsset', record.uuid);
              console.log(`*** Result: ${prettyJSONString(result.toString())}`);
  
              console.log('\n--> Evaluate Transaction: AssetExists, function returns "true" if an asset with given assetID exist');
              result = await contract.evaluateTransaction('AssetExists', record.uuid);
              console.log(`*** Result: ${prettyJSONString(result.toString())}`);
  
              console.log('\n--> Evaluate Transaction: AssetValid, function returns "true" if an asset with given assetID exist');
              result = await contract.evaluateTransaction('AssetValid', record.uuid);
              console.log(`*** Result: ${prettyJSONString(result.toString())}`);
  
              
          } finally {
              // Disconnect from the gateway when the application is closing
              // This will close all connections to the network
              gateway.disconnect();
          }
      } catch (error) {
          console.error(`******** FAILED to run the application: ${error}`);
      }
  }
  exports.submit = submit;
  
  /**
    * @param record A record produced using the body of the request in the check function
    */
   async function checkRecord(record) {
      try {
          // build an in memory object with the network configuration (also known as a connection profile)
          const ccp = buildCCPOrg2();
  
          // build an instance of the fabric ca services client based on
          // the information in the network configuration
          const caClient = buildCAClient(FabricCAServices, ccp, 'ca.org2.example.com');
  
          // setup the wallet to hold the credentials of the application user
          const wallet = await buildWallet(Wallets, walletPath);
  
          // in a real application this would be done on an administrative flow, and only once
          await enrollAdmin(caClient, wallet, mspOrg2);
  
          // in a real application this would be done only when a new user was required to be added
          // and would be part of an administrative flow
          await registerAndEnrollUser(caClient, wallet, mspOrg2, org2UserId, 'org2.department1');
  
          // Create a new gateway instance for interacting with the fabric network.
          // In a real application this would be done as the backend server session is setup for
          // a user that has been verified.
          const gateway = new Gateway();
  
          try {
              // setup the gateway instance
              // The user will now be able to create connections to the fabric network and be able to
              // submit transactions and query. All transactions submitted by this gateway will be
              // signed by this user using the credentials stored in the wallet.
              await gateway.connect(ccp, {
                  wallet,
                  identity: org2UserId,
                  discovery: { enabled: true, asLocalhost: true } // using asLocalhost as this gateway is using a fabric network deployed locally
              });
  
              // Build a network instance based on the channel where the smart contract is deployed
              const network = await gateway.getNetwork(channelName);
  
              // Get the contract from the network.
              const contract = network.getContract(chaincodeName);
  
              // Initialize a set of asset data on the channel using the chaincode 'InitLedger' function.
              // This type of transaction would only be run once by an application the first time it was started after it
              // deployed the first time. Any updates to the chaincode deployed later would likely not need to run
              // an "init" type function.
              console.log('\n--> Submit Transaction: InitLedger, function creates the initial set of assets on the ledger');
              await contract.submitTransaction('InitLedger');
              console.log('*** Result: committed');
  
              // Let's try a query type operation (function).
              // This will be sent to just one peer and the results will be shown.
              console.log('\n--> Evaluate Transaction: GetAllAssets, function returns all the current assets on the ledger');
              let result = await contract.evaluateTransaction('GetAllAssets');
              console.log(`*** Result: ${prettyJSONString(result.toString())}`);
  
              console.log('\n--> Evaluate Transaction: ReadAsset, function returns an asset with a given assetID');
              result = await contract.evaluateTransaction('ReadAsset', record.uuid);
              console.log(`*** Result: ${prettyJSONString(result.toString())}`);
  
              console.log('\n--> Evaluate Transaction: AssetExists, function returns "true" if an asset with given assetID exist');
              result = await contract.evaluateTransaction('AssetExists', record.uuid);
              console.log(`*** Result: ${prettyJSONString(result.toString())}`);
  
              console.log('\n--> Evaluate Transaction: AssetValid, function returns "true" if an asset with given assetID exist');
              result = await contract.evaluateTransaction('AssetValid', record.uuid);
              console.log(`*** Result: ${prettyJSONString(result.toString())}`);
              
          } finally {
              // Disconnect from the gateway when the application is closing
              // This will close all connections to the network
              gateway.disconnect();
          }
      } catch (error) {
          console.error(`******** FAILED to run the application: ${error}`);
      }
  }
  exports.checkRecord = checkRecord;
  
  
  exports.index = function (req, res) {
      res.sendFile(path.resolve('views/records.html'));
  };
  
  exports.check = function (req, res) {
      hash.update(req.body.uuid);
      var hashedRec = new Record({uuid : hash.digest('hex')});
      var newRecord = new Record(req.body);
      Record.find(hashedRec).exec(function (err) {
          if (err) {
              res.status(400).send('Unable to find record in database');
          } else {
              checkRecord(newRecord);
          }
      });
  };
  
  exports.create = function (req, res) {
      hash.update(req.body.uuid);
      var hashedRec = new Record({uuid : hash.digest('hex')});
      var newRecord = new Record(req.body);
      hashedRec.save(function (err) {
          if (err) {
              res.status(400).send('Unable to save record to database');
          } else {
              submit(newRecord);
          }
      });
  };
  
  exports.createuuid = function (req, res) {
      var uuid = uuid4();
      res.send(uuid);
  };
\end{lstlisting}

\subsection{Model}
Code for the model that handles data formatting, for this Mongoose is used as the ORM for MongoDB. \label{appendix:model}
\begin{lstlisting}
  const mongoose = require('mongoose');
  const Schema = mongoose.Schema;
  
  const Record = new Schema ({
          uuid: { type: String, required: true },
          timestamp: { type: Date, required: false },
          owner: { type: String, required: false },
          exp: { type: Date, required: false },
  });
  
  module.exports = mongoose.model('Record', Record)
  
\end{lstlisting}

\section{Chaincode} 
The chaincode developed for the DIIS, representing immunisation records. \footnote{Adapted from tutorial provided in official documentation for Hyperledger Fabric "https://hyperledger-fabric.readthedocs.io/en/release-2.2/chaincode4ade.html"} \label{appendix:chaincode}
\begin{lstlisting}[language=Go, caption={Chaincode representing immunisation records.}]
  package main

  import (
    "encoding/json"
    "fmt"
    "log"
    "strconv"
  
    //"time"
  
    "github.com/hyperledger/fabric-contract-api-go/contractapi"
  )
  
  // Record provides functions for managing an Asset
  type Record struct {
    contractapi.Contract
  }
  
  // medical records should contain metadata of a patient-provider encounter (visit date/time, location, etc.),
  
  // Asset describes basic details of what makes up a simple asset
  type Asset struct {
    UUID       string `json:"uuid"`
    Timestamp  string `json:"dateTime"`
    Owner      string `json:"owner"`
    Expiration string `json:"expiration"`
  }
  
  // InitLedger adds a base set of assets to the ledger
  func (s *Record) InitLedger(ctx contractapi.TransactionContextInterface) error {
    assets := []Asset{
      {UUID: "0a970bbe-b436-4601-87d2-becd3bf84054", Timestamp: "1511382400000", Owner: "Jane Doe", Expiration: "1655510400000"}, // add an attribute for Immunisation, representing the disesase + variant?
      {UUID: "f57d594d-3a88-4fcc-80a2-4147d23923e3", Timestamp: "1721382400000", Owner: "Ajay Singh", Expiration: "1655510400000"},
      {UUID: "6d8670a9-8550-40b9-9c57-30bb33a4d6f4", Timestamp: "1621382402100", Owner: "Zhang San", Expiration: "1655510400000"},
      {UUID: "3519bdc7-eb13-4ac0-bd63-137241af313b", Timestamp: "1621382448000", Owner: "Max Mustermann", Expiration: "1655510400000"},
      {UUID: "183ce652-0788-4ea7-896b-93d6036db83e", Timestamp: "1611382409500", Owner: "Pierre Paul", Expiration: "1655510400000"},
      {UUID: "8e0c5ba3-4dfe-41fc-b454-26a3a276ac92", Timestamp: "1511382400120", Owner: "Wang Wu", Expiration: "1655510400000"},
    }
  
    for _, asset := range assets {
      assetJSON, err := json.Marshal(asset)
      if err != nil {
        return err
      }
  
      err = ctx.GetStub().PutState(asset.UUID, assetJSON)
      if err != nil {
        return fmt.Errorf("failed to put to world state. %v", err)
      }
    }
  
    return nil
  }
  
  // CreateAsset issues a new asset to the world state with given details.
  func (s *Record) CreateAsset(ctx contractapi.TransactionContextInterface, uuid string, timestamp string, owner string, expiration string) error {
    exists, err := s.AssetExists(ctx, uuid)
    if err != nil {
      return err
    }
    if exists {
      return fmt.Errorf("the asset %s already exists", uuid)
    }
  
    asset := Asset{
      UUID:       uuid,
      Timestamp:  timestamp,
      Owner:      owner,
      Expiration: expiration,
    }
    assetJSON, err := json.Marshal(asset)
    if err != nil {
      return err
    }
  
    return ctx.GetStub().PutState(uuid, assetJSON)
  }
  
  // ReadAsset returns the asset stored in the world state with given uuid.
  func (s *Record) ReadAsset(ctx contractapi.TransactionContextInterface, uuid string) (*Asset, error) {
    assetJSON, err := ctx.GetStub().GetState(uuid)
    if err != nil {
      return nil, fmt.Errorf("failed to read from world state: %v", err)
    }
    if assetJSON == nil {
      return nil, fmt.Errorf("the asset %s does not exist", uuid)
    }
  
    var asset Asset
    err = json.Unmarshal(assetJSON, &asset)
    if err != nil {
      return nil, err
    }
  
    return &asset, nil
  }
  
  // DeleteAsset deletes an given asset from the world state.
  func (s *Record) DeleteAsset(ctx contractapi.TransactionContextInterface, uuid string) error {
    exists, err := s.AssetExists(ctx, uuid)
    if err != nil {
      return err
    }
    if !exists {
      return fmt.Errorf("the asset %s does not exist", uuid)
    }
  
    return ctx.GetStub().DelState(uuid)
  }
  
  // AssetExists returns true when asset with given UUID exists in world state
  func (s *Record) AssetExists(ctx contractapi.TransactionContextInterface, uuid string) (bool, error) {
    assetJSON, err := ctx.GetStub().GetState(uuid)
    if err != nil {
      return false, fmt.Errorf("failed to read from world state: %v", err)
    }
  
    return assetJSON != nil, nil
  }
  
  // AssetValid returns true when an asset's timestamp is less than the expiration
  func (s *Record) AssetValid(ctx contractapi.TransactionContextInterface, uuid string) (bool, error) {
    assetJSON, err := ctx.GetStub().GetState(uuid)
    if err != nil {
      return false, fmt.Errorf("failed to read from world state: %v", err)
    }
  
    var asset Asset
    json.Unmarshal(assetJSON, &asset)
  
    var stamp, error = strconv.ParseUint(asset.Timestamp, 10, 64)
    if error != nil {
      return false, fmt.Errorf("failed to Validate asset %v", error)
    }
  
    var exp, anotherErr = strconv.ParseUint(asset.Expiration, 10, 64)
    if anotherErr != nil {
      return false, fmt.Errorf("failed to Validate asset %v", anotherErr)
    }
  
    return stamp < exp, nil
  }
  
  // TransferAsset updates the owner field of asset with given uuid in world state.
  func (s *Record) TransferAsset(ctx contractapi.TransactionContextInterface, uuid string, newOwner string) error {
    asset, err := s.ReadAsset(ctx, uuid)
    if err != nil {
      return err
    }
  
    asset.Owner = newOwner
    assetJSON, err := json.Marshal(asset)
    if err != nil {
      return err
    }
  
    return ctx.GetStub().PutState(uuid, assetJSON)
  }
  
  // GetAllAssets returns all assets found in world state
  func (s *Record) GetAllAssets(ctx contractapi.TransactionContextInterface) ([]*Asset, error) {
    // range query with empty string for startKey and endKey does an
    // open-ended query of all assets in the chaincode namespace.
    resultsIterator, err := ctx.GetStub().GetStateByRange("", "")
    if err != nil {
      return nil, err
    }
    defer resultsIterator.Close()
  
    var assets []*Asset
    for resultsIterator.HasNext() {
      queryResponse, err := resultsIterator.Next()
      if err != nil {
        return nil, err
      }
  
      var asset Asset
      err = json.Unmarshal(queryResponse.Value, &asset)
      if err != nil {
        return nil, err
      }
      assets = append(assets, &asset)
    }
  
    return assets, nil
  }
  
  func main() {
    assetChaincode, err := contractapi.NewChaincode(&Record{})
    if err != nil {
      log.Panicf("Error creating record chaincode: %v", err)
    }
  
    if err := assetChaincode.Start(); err != nil {
      log.Panicf("Error starting record chaincode: %v", err)
    }
  }
  
\end{lstlisting}

\section{Containerisation}
Here are the files used to containerise the web application and database. \label{appendix:containerisation}
\subsection{Dockerfile}
\begin{lstlisting}
  FROM node:latest

  # LABEL maintainer="emailkaine@gmail.com"
  
  # Create app directory
  WORKDIR /usr/src/app
  
  #COPY test-network ./
  
  #COPY chaincode ./
  
  COPY package*.json ./
  
  RUN npm install
  
  # Bundle app source
  COPY . .
  
  EXPOSE 3000
  
  
  
  ENTRYPOINT [ "node", "app.js" ]
\end{lstlisting}

\subsection{docker-compose.yaml}
\begin{lstlisting}
  version: '3'

  services:
    nodejs:
      build:
        context: .
        dockerfile: Dockerfile
      image: nodejs
      container_name: nodejs
      env_file: .env
      environment:
        - MONGO_USERNAME=$MONGO_USERNAME
        - MONGO_PASSWORD=$MONGO_PASSWORD
        - MONGO_HOSTNAME=db
        - MONGO_PORT=$MONGO_PORT
        - MONGO_DB=$MONGO_DB
      ports:
        - "3000:3000"
      depends_on: 
          db:
              condition: service_healthy
      volumes:
        - .:/home/node/app
        - node_modules:/home/node/app/node_modules
      networks:
        - app-network
      command: ENTRYPOINT [ "node", "app.js" ]
  
    db:
      image: healthcheck/mongo
      container_name: db
      restart: unless-stopped
      env_file: .env
      environment:
        - MONGO_INITDB_ROOT_USERNAME=$MONGO_USERNAME
        - MONGO_INITDB_ROOT_PASSWORD=$MONGO_PASSWORD
      volumes:
        - dbdata:/data/db
      networks:
        - app-network
  
  networks:
    app-network:
      driver: bridge
  
  volumes:
    dbdata:
    node_modules:
\end{lstlisting}