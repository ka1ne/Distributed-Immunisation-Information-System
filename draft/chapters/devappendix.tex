\section{Docker}
Used to containerise the web application and database for the system, as well as containers from Hyperledger Fabric images. 
Containers package up all code and necessary dependencies so applications can run reliably on differing systems. \footnote{https://docs.docker.com/}

\section{Docker Compose}
A tool used to define and run multi-container Docker applications. YAML files are used to configure application's services. 
Used in this project to build the web application and database together. \footnote{https://docs.docker.com/compose/}

\section{PlantUML}
An open-source tool that allows users to create diagrams from text. 
Used to generate diagrams throughout this report, using both PNGs generated on the PlantUML web server \footnote{http://www.plantuml.com/plantuml} and the tex package \footnote{https://ctan.org/pkg/plantuml?lang=en} used with Latex. \label{appendix:plantuml}

\section{Diagram Sprites}
Repository of sprites, used with PlantUML to easily identify technologies in diagrams. \footnote{https://github.com/tupadr3/plantuml-icon-font-sprites} \label{appendix:sprites}

\section{C4-PlantUML}
A tool to facilitate building C4 model diagrams in PlantUML. \footnote{https://github.com/plantuml-stdlib/C4-PlantUML} \label{appendix:c4plantuml}

\section{ExpressJS}
A web framework for Nodejs, provides HTTP utility methods and middleware to aid backend development. \footnote{https://expressjs.com/}

\section{EJS}
A simple templating language to generate HTML markup using JavaScript. \footnote{https://ejs.co/}
\label{appendix:ejs}

\section{MongoDB}
A general purpose, document-based, database for web applications. \footnote{https://www.mongodb.com/}

\section{MongooseJS}
An Object Data Modelling (ODM) library for MongoDB. It provides a schema-based method of modelling application data. \footnote{https://mongoosejs.com/}

\section{UUID Node.js}
A package for creating \href{https://datatracker.ietf.org/doc/html/rfc2986}{RFC4122} UUIDs. \footnote{https://www.npmjs.com/package/uuid} \label{appendix:uuid}

\section{Crypto}
Crypto module, provides cryptographic functionality, used for hashing data before entering in database. \footnote{https://nodejs.org/api/crypto.html} \label{appendix:crypto}